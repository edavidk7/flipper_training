\documentclass[a4paper,12pt]{article}

\usepackage{tocloft}
\setlength{\cftbeforesecskip}{0.2cm}
% Enter the inputs below
%-------------------------------------------------------------------------------
\newcommand{\studentName}{David Korčák}
\newcommand{\projectTitle}{GPU-based simulation and reinforcement learning pipeline for tracked ground robots}
\newcommand{\academicYear}{Prague, January 2025}
\newcommand{\docType}{B4BPROJ6 report}
\newcommand{\studyProgramme}{Open Informatics}
\newcommand{\branchOfStudy}{Artificial Intelligence and Computer Science}
\newcommand{\supervisorName}{doc. Ing. Karel Zimmermann, Ph.D.}
%-------------------------------------------------------------------------------

% Style file (.sty) with dissertation format
\usepackage{report_template}


\begin{document}


% TITLE AND DECLARATION PAGESFluids_MSc_DissertationFluids_MSc_Dissertation
%------------------------------------------------------------------------
\frontMatter
%------------------------------------------------------------------------


% ABSTRACT
%------------------------------------------------------------------------
\section*{Abbreviations}
\label{sec:abbreviations}
\addcontentsline{toc}{section}{Abbreviations}

\textbf{The abstract must not exceed one page of A4.} The abstract must be self-contained and the main results should be clearly summarised. \textbf{Keep the text in the third person.} If you use abbreviations you should spell it out first in full, e.g. Finite Element Method (FEM), etc. The abstract should not contain references.
%------------------------------------------------------------------------


\clearpage
\pagestyle{fancy}


% TABLE OF CONTENTS
%------------------------------------------------------------------------
% \vspace*{-2cm} % adjust the spacing as needed
\tableofcontents % Use \thispagestyle for correct headers and footers
%------------------------------------------------------------------------


\clearpage

% INTRODUCTION
%------------------------------------------------------------------------
\section{Introduction}
\label{sec:introduction}

This is an extremely important section. A good introduction should leave the reader with a clear idea of the problem to be tackled and looking forward to the more detailed chapters to follow. An essential part of the introduction is to define clearly the overall aim and objectives of the research project and to describe where these are addressed in the report.

Below you can find some examples of sub-sections.



\clearpage


% BACKGROUND
%------------------------------------------------------------------------
\section{Background}
\label{sec:background}

This should consist of a relevant background to the subject. This will involve a literature survey, which need not be exhaustive but should be critical and concentrate on the most relevant, (possibly controversial aspects) to the work described in the report. It should not occupy a large part of the dissertation, unless it in some way forms the core of the work. The primary objective of the literature review is to establish the ‘state-of-science’ and to demonstrate, through critique of existing work, why the approaches to be outlined subsequently are worth pursuing.

\textbf{All equations must be numbered}. For example here is the 2D mass continuity equation for incompressible fluids


\begin{equation}
  \label{eq:mass-continuity}
  \frac{\partial u}{\partial x} + \frac{\partial w}{\partial z} = 0
\end{equation}

and here is the equation for irrotational 2D flows

\begin{equation}
  \label{eq:irrotationality}
  \frac{\partial w}{\partial x} - \frac{\partial u}{\partial z} = 0
\end{equation}
%------------------------------------------------------------------------


\clearpage


% METHODOLOGY
%------------------------------------------------------------------------
\section{Methodology}
\label{sec:methodology}


This should describe the general basis of the research approach – such as the form of hypotheses, general methods of statistical analysis, analytical framework, experimental procedures, numerical setup, etc... Your methodology needs to be adequately described and its use fully justified.
%------------------------------------------------------------------------


\clearpage


% RESULTS AND DISCUSSION
%------------------------------------------------------------------------
\section{Results and Discussion}
\label{sec:results-discussion}

Describe results in a logical order, which is not necessarily the order in which you performed the project. It is a common failing to assume that the reader knows the project as well as you do. He or she must not be expected to use the arts of a detective to find and decipher the important information. Therefore, refer appropriately to figures or tables and remember to emphasise and perhaps quote significant results. Do not deluge the reader with data and figures; use appendices for details if necessary and concentrate on key results in the body of the text. Briefly summarise the main results at the end of each main sequence of experiments/simulations.

Analysis of your results should include, for example, clear graphical representations, and, when necessary, appropriate statistical analysis.

The Discussion part should attempt to tie together the results and what they indicate in a broader
context, including discussion in relation to the literature and implications beyond the immediate
confines of your specific project. It should also illustrate the extent to which the original aims
have been satisfied, any difficulties you have identified and what future work is suggested.
\textbf{Great care is needed with this with this part of the dissertation.} 

\clearpage

Here is an example of how to use figures within the dissertation:

\begin{quote}
  Figure~\ref{fig:wave_breaking_lab_measurements} illustrates the breaker index as a function of the
  Irribarren number $N_I$, as found by the experimental investigations undertaken by
  \citet{Battjes1974}.
\end{quote}

\begin{figure}[!h]
  \centering
  \includegraphics[width=0.6\textwidth]{fig/wave_breaking_lab_measurements.png}
  \caption{Breaker index, $\gamma$, versus Irribarren number, $N_I$, fit to experimental measurements \citep{Battjes1974}.}
  \label{fig:wave_breaking_lab_measurements}
\end{figure}

Here is an example of using tables within the dissertation:

\begin{quote}
  The wave conditions within the coastal zone are dependent on the water depth regime in which the
  waves propagate. Table~\ref{tab:water-depth-regimes} presents the values of the relative water
  depth and group velocity for each water depth regime.
\end{quote}

\begin{table}[!h] 
  \centering
    \begin{tabular}{| l | c | c |}
     \bf Regime              &  \bf Relative water depth           &  \bf Group velocity        \\
        \hline
         Deep water          &  $kd > 4$                       &  $c_g = 0.5c$  \\
         Intermediate water  &  $0.3\leqslant kd \leqslant 4$    &  $c_g=\frac{c}{2}\left( 1 + \frac{2kd}{\sinh{(2kd)}}\right)$  \\
         Shallow water       &  $kd < 0.3$                     &  $c_g = c$\\
    \end{tabular}
    \caption{The relative water depth, $kd$, and group velocity, $c_g$, in terms of the phase velocity, $c$, for the various water depth regimes.}
    \label{tab:water-depth-regimes}
\end{table}
%------------------------------------------------------------------------

\clearpage


% CONCLUSIONS
%------------------------------------------------------------------------
\section{Conclusions}
\label{sec:conclusions}


This chapter is important to summarise your overall conclusions. It should be brief, with the final
conclusions and/or recommendations clearly identifiable without having to search for them. The use
of bullet points might help if you have a large number of conclusions. It should not simply repeat
what has been discussed in the preceding sections, but should summarise. It should not contain new information. It should then draw wider conclusions, including implications beyond the immediate case study or survey. The effective drawing out of these wider conclusions is what often marks out an excellent dissertation from a good or average dissertation. How does your research fit in the context of the existing literature you will already have reviewed: is it consistent or contradictory; have you found out anything `new'? This is also where you should clearly state any recommendations for future research.

This section, however, is not the only place where conclusions can be drawn and presented.  Individual sections are likely to have their own summaries and conclusions that will lead into the more comprehensive discussion section and a final summary and conclusions section.
%------------------------------------------------------------------------


\clearpage

% Data, code and use of Generative AI tools (e.g. ChatGPT)
%------------------------------------------------------------------------
\section{Use of Third-party Data, Code and Generative AI Tools}
\label{sec:DCG_Usaage}

This section is where you are required to acknowledge and report the source of any third-party data and code used in your work. For example, you may have used data collected in a previous experiment (physical or numerical) and you may have used code written by others to process those data. In addition, if you have used Generative Artificial Intelligence tools (e.g. ChatGPT) in anyway you are required to state how you have used these tools.

This section does not count towards your 50-page upper limit.

\clearpage


% REFERENCES
%------------------------------------------------------------------------
\bibliographystyle{Fluids_MSc_Dissertation}
\bibliography{references}
\addcontentsline{toc}{section}{References} %Add this to the contents page

\vspace*{10mm}

\textbf{You will be penalised for incorrect referencing in your final dissertation.} Cite all references, with the name of the author(s) and year of publication in brackets. More than
two authors are quoted as first author \textit{et al.} followed by the year.  Two or more publications within
the same year by the same authors should be distinguished by the letters a, b, c, etc... Collect all
references together at the end of the report and list them alphabetically in standard form; \textbf{an
example of this is given above.} The \textbf{Harvard system} should be used and examples of it’s use in the body of text are given below:


\begin{quote}
This procedure is widely used \citep{Christou2009,Wu2016,Hughes2016a,Hughes2016b}. \citet{Christou2014} state that.... \citet{Spinneken2009a,Spinneken2009b} concluded that...
\end{quote}


Please note that it is important to have consistency in the style you use throughout your dissertation. \textbf{List only the references you cite in the text.} Any references not seen in the original language should be marked with an asterisk (*). If only seen as a translation, follow references with ``English translation''. Websites should be referenced in the main body of the text as other references (by author or organisation), with the final reference list containing details of the full URL the authority or organisation concerned, and the date at which it was accessed (since websites may change and may not be able to be accessed subsequently). You should avoid relying too heavily on websites – try to reinforce your referencing with sufficient academic literature. Journals accessed electronically are still journals and quoted as normal.
%------------------------------------------------------------------------


\clearpage


% APPENDICES
%------------------------------------------------------------------------
\appendix

\section{Appendix Title}
\label{sec:appendix}

Label your appendices alphabetically: A, B, C, etc... using the Heading Appendix style. Add as many as you require, however please make sure that their content is relevant to report and not just a dump of every single thing that you did during your project. You may be penalised if the Appendix is excessive and not relevant, e.g. if appendices appear unconnected with the main text.

\textbf{Under no circumstances should important material be included in the Appendix, instead of the
  main text, in order to stay within the word limit.}

Large sets of relevant data (e.g., census results, "raw" experimental results) should be presented in an Appendix. Any useful parts of the study not directly relevant to the main theme may also be put in an Appendix, but should be clearly referenced in the text. 

Computer programs: if the program has been published, give a reference to it and a brief outline of the methods it uses. If the program is original give a listing as an Appendix preceded by a single page description of the program or on a data stick, as recommended by your supervisor.
%------------------------------------------------------------------------


\end{document}



